\documentclass[12pt]{article}
\usepackage[T1]{fontenc}
\usepackage[utf8]{inputenc}
\usepackage{pdflscape}
\usepackage[margin=2cm, left=3cm]{geometry}
\usepackage{color}
\usepackage{amsmath}
\usepackage{amssymb}
\usepackage{tocbasic}
\usepackage{fancyhdr}
\usepackage{float} 
\usepackage{graphicx}
\usepackage{subcaption}
\usepackage[english, ngerman]{babel}
\usepackage{hyperref}
\usepackage[
    backend=biber,		
    bibwarn=true,
    bibencoding=utf8,	% wenn .bib in utf8, sonst ascii
    sortlocale=de_DE,
    style=numeric-comp,
]{biblatex}
\usepackage{csquotes}

\addbibresource{bibliographie.bib}

\selectlanguage{ngerman}

\hypersetup{%
    colorlinks=true, 		% Aktivieren von farbigen Links im Dokument
    linkcolor=black, 	    % Farbe festlegen
    citecolor=black,
    filecolor=black,
    menucolor=black,
    urlcolor=black,
    linktoc=all,            % Seitenzahlen und Text klickbar
    bookmarksnumbered=true 	% Überschriftsnummerierung im PDF Inhalt anzeigen.
}

\newcommand\todo[1]{\textit{\textcolor{red}{\\TODO: #1}}}

\DeclareNewTOC[%
 forcenames,
 type=formel,
 name={Formel},%
 listname={Formelverzeichnis}
]{for}
\newcommand*{\formelentry}[1]{%
     \addcontentsline{for}{formel}{\protect\numberline{\theequation} #1}%
    }



\title{Master Autonomes Fahren - Mathematik Zusammenfassung}
\author{Marcel Wagner}
\date{\today}


\begin{document}
\pagenumbering{Roman}
\maketitle
\tableofcontents
\newpage
\listoffigures
%\listoffigures
\listofformels
\newpage
\pagenumbering{arabic}
\setcounter{page}{1}
\section{Mathematische Symbole}
\subsection{Mengen}
\begin{tabular}{p{0.1\textwidth}|p{0.2\textwidth}|p{0.6\textwidth}}
	Symbol & Verwendung & Bedeutung\\
	\hline
	$\in$ & $\omega\quad\in\quad\Omega$ & Element ($\omega$ ist in $\Omega$ enthalten) \\
	$\cap$ & $A\quad\cap\quad B$ & Disjunkt (Kein Teil von A ist ein Teil von B) \\
	$\cup$ & $A\quad\cup\quad B$ & Kunjunktion (Ein Teil von A ist ein Teil von B)\\
	$\subseteq$ & $A\quad\subseteq\quad B$ & Teilmenge (A ist eine Teilmenge von B)\\
	$\setminus$ & $A\quad\setminus\quad B$ & Differenz (Differenz der mengen A und B)\\
	$^\mathrm{C}$ & $A^\mathrm{C}$ & Komplement (Differenz des Universums (kann eine\newline größere Menge sein) und der Teilmenge)
\end{tabular}

\section{Statistik}
\subsection{Arithmetisches Mittel}
\begin{equation}
	\overline{x} := \frac{x_1 + ... + x_n}{n} = \frac{1}{n}\sum_{i=1}^{n}{x_i}
\end{equation}
\subsection{Mittlerer Abstand}
Der mittlere Abstand wird nicht sehr häufig verwendet, da das Rechnen mit Beträgen sehr mühsam ist. Die Varianz (durchschnittliche quadratische Abweichung) eignet sich besser.
\begin{equation}
	\frac{1}{n}\sum_{i=1}^{n}{|x_i-\overline{x}|}
\end{equation}
\subsection{Varianz}
\begin{equation}
	s_x^2= \frac{1}{n-1}\sum_{i=1}^n{(x_i-\overline{x})^2}
\end{equation}
\subsection{Standartabweichung}
\begin{equation}
	s_x = \sqrt{s_x^2}
\end{equation}
\subsection{Kovarianz}
\begin{equation}
	y_{xy}:=\frac{1}{n-1}\sum_{i=1}^n(x_i-\overline{x})(y_i-\overline{y})
\end{equation}
\subsection{Korrelationskoeffizient}
\begin{equation}
	r_{xy}:=\frac{s_{xy}}{s_x\cdot s_y}
\end{equation}
\subsection{Regressionsgerade}
\begin{equation}
	y = a + bx
\end{equation}
\begin{equation}
	b = \frac{s_{xy}}{s_x^2}
\end{equation}
\begin{equation}
	a = \overline{y} - b\overline{x}
\end{equation}
\subsection{Bestimmtheitsmaß}
\begin{equation}
	R^2 = \frac{\sum_{i=1}^n (\hat{y}_i-\overline{y})^2}{\sum_{i=1}^n (y_i-\overline{y})^2} = \frac{\sum_{i=1}^n (y_i - \hat{y}_i)^2}{\sum_{i=1}^n (y_i-\overline{y})^2}
\end{equation}
\todo{Beschreibung von y dach und y quer}
\subsection{Bestimmtheitsmaß}
\begin{equation}
	R^2 = r_{xy}^2
\end{equation}

\section{Wahrscheinlichkeitsrechnung}
\subsection{Fakultät}
\begin{equation}
	n! = n \cdot (n-1) \cdot \dotsc \cdot 2 \cdot 1 
\end{equation}
\subsection{Binomialkoeffizient}
\begin{equation}
	{n \choose k} = \frac{n!}{k!\cdot(n-k)!}
\end{equation}
\subsection{Kugeln Ziehen}
\begin{tabular}{c|c|c}
	 & mit Reihenfolge & ohne Reihenfolge 
	 \\\hline
	 mit Zurücklegen & $n^k$ & ${n+k-1 \choose k}$
	 \\\hline
	 ohne Zurücklegen & $\frac{n!}{(n-k)!}$ & ${n \choose k}$
\end{tabular}
\subsection{Menge}
Unter einer Menge verstehen wir die Zusammenfassung unterscheidbarer Elemente zu einer Gesamtheit.
\subsection{Gleichheit}
$A = B:\Leftrightarrow$ $A$ und $B$ besitzen die gleichen Elemente.
\subsection{Teilmenge}
$A \subset B:\Leftrightarrow$ wenn alle Elemente von $A$ auch in $B$ sind, dann ist $A$ eine Teilmenge von $B$ oder auch $B$ die Obermenge von $A$.\\
Jede Menge ist Teilmenge von sich selbst. 
\subsection{Potenzmenge}
Die Potenzmenge $\mathcal{P}(X)$ ist eine Menge welche aus allen Teilmengen von $U \subseteq X$ besteht.
\subsection{Mächtigkeit}
$|A|:=$ Zahl der Elemente von A.
\subsection{Vereinigung}
$A\cup B:=$ Menge aus allen Elementen welche in $A$ oder in $B$ oder in beiden enthalten sind.
\subsection{Schnitt}
$A\cap B:=$ Menge aus allen Elementen welche in $A$ und in $B$ enthalten sind.
\subsection{Differenz}
$A\setminus B:=$ Menge aus allen Elementen welche zu $A$ aber \textbf{nicht} zu $B$ gehören.
\subsection{Komplement}
$A^C:=$ Menge aus allen Elementen welche \textbf{nicht} zu $A$ gehören.
\subsection{Kartesisches Produkt}
\begin{equation}
	A \times B:={(a, b):a\in A, b\in B}
\end{equation}
\subsection{Zufallsexperiment}
\begin{itemize}
	\item Genau festgelegte Bedingungen
	\item Zufälliger Ausgang
	\item Beliebig oft wiederholbar
	\item Ein Versuch bezeichnet einen Vorgang bei dem mehrere Ergebnisse (Elementarereignis) eintreten können
	\item Menge aller Elementarereignisse wird als Ergebnismenge (Ergebnisraum) $\Omega$ bezeichnet
\end{itemize}
\subsection{Ereignis}
\begin{itemize}
	\item Eine Teilmenge $A \subset \Omega$ heißt Ereignis
	\item $A = \emptyset$ unmögliches Ereignis
	\item $A = \Omega$ sicheres Ereignis
\end{itemize}
\subsection{Disjunkte Ereignisse}
Zwei ereignisse sind disjunkt (unvereinbar) wenn deren Schnitt gleich der leeren Menge ist $A \cap B = \emptyset$.

\subsection{$\sigma$-Algebra}
Eine Teilmenge einer Potenzmenge (Menge von Teilmengen, $\mathcal{A}\subseteq\mathcal{P}(\Omega)$) heißt $\sigma$-Algebra wenn sie folgende Bedingungen erfüllt:
\begin{itemize}
	\item Die Teilmenge $\mathcal{A}$ der Potenzmenge $\mathcal{P}(\Omega)$ enthält die Grundmenge $\Omega$.
	\item Das Komplement $A^\mathrm{C}$ eines Elements der Teilmenge $A \in\mathcal{A}$ ist gleich der Differenz aus Grundmenge und Element $A^\mathrm{C} := \Omega\setminus A$. Stabilität des Komplements.
	\item Sind die Mengen in der Teilmenge der Potenzmenge $A_1,A_2,A_3,... \in \mathcal{A}$ enthalten, so ist auch die Vereinigung aller Mengen in der Teilmenge der Potenzmenge enthalten $\bigcup\limits_{n\epsilon \mathbb{N}}A_n \in \mathcal{A}$
	\item Alle vorangegangenen Mengenoperationen können auf die Teilmengen angewendet werden. 
\end{itemize}
\subsection{Axiome der Wahrscheinlichkeitsrechnung}
Die Funktion $P$ ordnet jedem Ereignis $A$ eine Wahrscheinlichkeit $P(A)$ zu.
\begin{enumerate}
	\item[(I)] Für jedes Ereignis $A\subset \Omega$ gilt $0\leq P(A) \leq 1$
	\item[(I')] Für das unmögliche Ereignis gilt $P(\emptyset) = 0$
	\item[(II)] Für das sichere Ereignis $\Omega$ gilt $P(\Omega) = 1$
	\item[(II')] Für ein Ereignis $A \subset \Omega$ gilt $P(A^C) = 1- P(A)$
	\item[(III)] Für disjunkte Ereignisse $A$ und $B$ gilt $P(A\cup B) = P(A)+P(B)$ 
	\item[(III')] Für zwei Ereignisse $A,B \subset \Omega$ gilt $P(A\cup B)= P(A)+P(B)-P(A\cap B)$ 
\end{enumerate}
\subsection{Laplace Experiment}
Endlich viele Elementarereignisse welche alle gleich wahrscheinlich sind.\\
Satz von Laplace:
\begin{equation}
	P(A) = \frac{|A|}{|\Omega|}=\frac{\text{Anzahl der Elementarereignisse in $A$}}{\text{Anzahl aller möglichen Elementarereignisse}}
\end{equation}
\subsection{Unabhängige Ereignisse}
Zwei Ereignisse heißen \textbf{unabhängig} wenn gilt: 
\begin{equation}
	P(A\cap B) = P(A) \cdot P(B)
\end{equation}
Sie heißen \textbf{abhängig} wenn sie nicht unabhängig sind.\\
Für unabhängige Ereignisse gilt:
\begin{equation}
	P(A) = \frac{P(A\cap B)}{P(B)} \text{\quad bzw. \quad} P(B)=\frac{P(A\cap B)}{P(A)}
\end{equation} 
\subsection{Bedingte Wahrscheinlichkeit}
"Wahrscheinlichkeit von A gegeben B".
\begin{equation}
	P(A|B) := \frac{P(A\cap B)}{P(B)}
\end{equation}
Sind $A, B \subset \Omega$ \textbf{unabhängige} Ereignisse gilt:
\begin{equation}
	P(A|B) = P(A)
\end{equation}
Sind $A, B \subset \Omega$ \textbf{abhängige} Ereignisse gilt:
\begin{equation}
	P(A|B) \neq P(A)
\end{equation}
\subsection{Multiplikationssatz}
\begin{equation}
	P(A\cap B) = P(A) \cdot P(B|A) = P(B) \cdot P(A|B)
\end{equation}
\subsection{Satz der totalen Wahrscheinlichkeit}
Der Ergebnisraum ist gegeben durch $\Omega = \bigcup\limits_{j=1}^\infty B_j$ mit $P(B_j)>0$ und alle $j$ sind paarweise Disjunkt $B_i \cap B_j = \emptyset$ für $i\neq j$
\begin{equation}
	P(A) = \sum_{j=1}^\infty P(A|B_j) \cdot P(B_j)
\end{equation}
Für den Spezialfall $\Omega = B\cup B^C$ gilt:
\begin{equation}
	P(A) = P(B) \cdot P(A|B) + P(B^C) \cdot P(A|B^C)
\end{equation}
\subsection{Satz von Bayes}
Besteht aus dem Multiplikationssatz \& der totalen Wahrscheinlichkeit:
\begin{equation}
	P(A|B)=\frac{P(A)\cdot P(B|A)}{P(B)}= \frac{P(A)\cdot P(B|A)}{P(A)\cdot P(B|A) + P(A^C)\cdot P(B|A^C)}
\end{equation}
\subsection{Zufallsvariablen}
Eine Zufallsvariable ist eine Abbildung des Ergebnisraums auf den reellen Zahlenraum $\Omega\longmapsto\mathbb{R}$. Die Zufallsvariable ordnet jedem Elementarereignis eine reelle Zahl zu.\\
Zwei Zufallsvariablen sind \textbf{unabhängig} wenn gilt:
\begin{equation}
	P(X\in A, Y\in B) = P(X\in A) \cdot P(Y\in B) \text{\quad für alle\quad} A,B\subset \mathbb{R} 
\end{equation}
Die Zufallsvariablen heißen \textbf{abhängig} wenn sie nicht unabhängig sind.\\
Die Zufallsvariable wird \textbf{diskret} genannt wenn sie nur endlich viele oder abzählbar unendlich viele Werte annimmt. Es gilt:
\begin{equation}
	\sum\limits_{i=1}^\infty P(X=x_i)=1
\end{equation}
\subsection{Wahrscheinlichkeitsfunktion}
Für die diskrete Zufallsvariable $X$ und ihre Ausprägungen lautet die Wahrscheinlichkeitsfunktion:
\begin{equation}
	p_X(x):=
	\begin{cases}
		P(X=x_i) \text{, für } x = x_i \text{ mit Zählindex } i \in \mathbb{N}\\
		0 \text{, sonst}
	\end{cases}
\end{equation}
\begin{equation}
	\sum\limits_{x_i}p_X(x_i)=1=p(\Omega)
\end{equation}
\subsection{Verteilungsfunktion diskreter Zufallsvariablen}
Für die diskrete Zufallsvariable $X$ und ihre Ausprägungen lautet die Verteilungsfunktion:
\begin{equation}
	F_X(x):=P(X\leq x)=\sum_{x_i\leq x}P(X=x_i)=\sum_{x_i\leq x}p_X(x_i)
\end{equation}
\subsection{Stetige Zufallsvariable}

\section{Zusatz}
\subsection{Integration}
\begin{equation}
	\int\limits_{a}^{b}f(x) dx = [F(x)]_a^b = F(b) - F(a)
\end{equation}
\begin{equation}
	\int\limits_{a}^{b}f(x) + g(x) dx = \int\limits_{a}^{b}f(x) dx + \int\limits_{a}^{b}g(x) dx
\end{equation}
\begin{equation}
	\int\limits_{a}^{b}c\cdot f(x) dx = c\cdot \int\limits_{a}^{b}f(x) dx
\end{equation}
\subsection{Partielle Integration}
\begin{equation}
	u(x)\cdot v(x) = \int u'(x) \cdot v(x) dx + \int u(x) \cdot v'(x) dx
\end{equation}
\todo{Basics Integration}
\subsection{Differentialgleichungen}
\todo{Basics DGL Lösungen}
\newpage
\pagenumbering{Alph}
\section{Anhang}
\printbibliography[heading=subbibnumbered]
\end{document}